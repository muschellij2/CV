\cventry{Mar 2019 -- Apr 2024}{Co-Investigator}{NIH/NCATS UL1TR003098 (Bandeen-Roche)}{}{Institutional Clinical and Translational Science Award}{The purpose of this application is to enhance both the process and benefits of clinical and translational research by bringing together the diverse resources of the Johns Hopkins Medical Institutions (JHMI) and creating a new model for carrying out scientific research.}{}


\cventry{Aug 2019 -- Jun 2024}{Co-Investigator}{NIH AWD00001055 (Tudorascu/Crainiceanu)}{}{Assist Dr. Tudorascu (main PI) with the development and implementation of methods for multi-sequence image segmentation and image intensity normalization in longitudinal studies. Drs. Muschelli and Crainiceanu will help mentor students involved in conducting the research and will support the development and deployment of open source R software on the Neuroconductor platform.}{}


\cventry{Sep 2019 -- Jul 2023}{Co-Investigator}{NIH/NIDA 1U54DA049110-01 (Lindquist)}{}{Data Center for Acute to Chronic Pain Biosignatures}{Understanding the mechanisms underlying the transition to chronic pain is a key to mitigating the dual epidemics of chronic pain and opioid use in the U.S. As part of the NIH Common Fund Acute to Chronic Pain Signatures (A2CPS) Program, we will establish a Data Integration and Resource Center (DIRC). The Center will integrate imaging, peripheral physiology, genomics and other omics, behavior, and clinical measures to develop biosignatures for the transition to chronic pain.}{}

\cventry{Sep 2018 -- Jun 2023}{Co-Investigator}{NIH/NHGRI U24HG010263-01 (Taylor/Schatz)}{}{Implementing the Genomic Data Science Analysis, Visualization, and Informatics Lab-space (AnVIL)}{We will develop the AnVIL environment using the leading supercomputing infrastructure as the foundation supporting the most widely used analysis environments and frameworks vetted by biomedical researchers. Our user-centered solution for data access, analysis, and visualization will enable investigators across all levels of expertise to fully utilize genomic datasets using environments they are already familiar with, leveraging well-engineered and optimized scientific computing infrastructure for greater efficiency and lower costs.}{}


\cventry{Jul 2020 -- Jun 2025}{Co-Investigator}{UE5CA254170 (MPI: Leek, Goecks, Watson, Wheelan)}{}{Scalable multi-mode education to increase use of ITCR tools by diverse analysts}{We propose to create a complete training resource including content and both online and offline courses to improve cancer informatics knowledge throughout the research enterprise. The project will create an informatics training network hosted at www.itcrtraining.org that can be used by everyone from community members, to basic scientists, to ITCR tool developers, to medical doctors, to principal investigators to improve their knowledge of informatics.}{}


