\cventry{2020--Present}{Associate Scientist}{Department of Biostatistics}{Johns Hopkins Bloomberg School of Public Health}{(Research-track Faculty)}{}
\vspace{1em}

\cventry{2016--2020}{Assistant Scientist}{Department of Biostatistics}{Johns Hopkins Bloomberg School of Public Health}{(Research-track Faculty)}{}
\vspace{1em}

\cventry{2012--2016}{Trainee}{T32AG021334: Epidemiology and Biostatistics of Aging Training Grant}{Mentors: Dr. Michelle Carlson, Dr. Ravi Varadhan}{}{}
\vspace{1em}

\cventry{2009--2016}{Research Associate}{Johns Hopkins Biostatistics Consulting Center}{Baltimore, MD}{}{
Collaborated on statistical projects with senior consultants.
\newline{} 
Weekly consulting for student research projects.
\newline{}
Report writing and analyzing data using statistical software: R, Stata.
}
\vspace{1em}

\cventry{2009--2014}{Data Analyst / Data Manager}{Brain Injury Outcomes Division}{Baltimore, MD}{}{
Decreased turnaround time on data safety report (from weeks to hours) by using knitr, LaTeX, and dynamic documents. \newline
Created a standardized database and processing pipeline for CT images. \newline
Analyzed phase II and III trials for treatment of intracerebral hemorrhage \newline{}
Data management and consultation of electronic case report form (eCRF) creation.}

\vspace{1em}
\cventry{2010--2012}{Data Analyst}{Laboratory for Neurocognitive and Imaging Research at Kennedy Krieger Institute}{Baltimore, MD}{}{
Reduced manual steps in complex imaging study analysis using automation from programming. \newline
Analysis of functional MRI (fMRI) imaging studies using Statistical Parametric Mapping. \newline{}
Programming consultant: Matlab \& R. }

%\vspace{1em}
%\cventry{2008}{Intern}{Analysis \& Inference}{Swarthmore, PA}{}{
%Cooperated on statistical projects and conferenced with clients about possible analysis options. \newline{}
%Report writing of analyses, data cleaning.}
%
%\vspace{1em}
%\cventry{2007}{Research Intern}{Dupont Stine-Haskell Laboratory}{Wilmington, DE}{}{Developed lab skills and techniques: cell culturing, making and sterilizing broth media, optical density readings, inoculations, quality control, cell counts, screening for fungicidal properties of compounds.
%}

